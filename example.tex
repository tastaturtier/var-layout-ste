\documentclass{var}
%\documentclass[ebook]{var}
\usepackage{private}
\usepackage{lipsum}
\begin{document}
\newcommand{\thisccnote}{S.~Leuchter, \textit{API-Technologien und -Design}, DOI:~\href{https://doi.org/10.5281/zenodo.XXXXXXX}{10.5281/zenodo.XXXXXXX}\\
\href{https://creativecommons.org/licenses/by-nc-sa/4.0/}{CC BY-NC-SA 4.0} 
Steinbeis-Edition (Verteilte Architekturen; 2, Stuttgart 2020}

\varpart{Teil\_A}{head2.png}{Bildnachweis von Head 2 (Teil A)}
\varchapter{Name von Kapitel\_1}{\thisccnote}
\lipsum[1-4] 
\begin{exkurs}{Ligaturentest}
    \lipsum[5]
    
    \textbf{Ligaturentests:}\index{ABC} \q{abc afin affin afln affln akke afte alia alla acke}
    
    \lipsum[6]Hier ist der Bildnachweis \q{Foto von Miraz092}\bildnachweis{Bildnachweis von Head 2 (Teil A)}
\end{exkurs}
\nomenclature{ABC}{Aah, Beh, Zeh}
\nomenclature{AXBC}{Aah, Beh, Zeh}
\nomenclature{ABC}{Aah, Beh, Zehx}
\nomenclature{AXBC}{Aah, Beh, Zexh}
\nomenclature{xABC}{Aah, Beh, Zeh}
\nomenclature{aAXBC}{Aah, Beh, Zeh}
\nomenclature{AyBC}{Aah, Beh, Zeh}
\nomenclature{AcvXBC}{Aah, Beh, Zeh}
\nomenclature{ABedC}{Aah, Beh, Zeh}
\nomenclature{AXBdfC}{Aah, Beh, Zeh}
\nomenclature{ABfC}{Aah, Beh, Zeh}
\nomenclature{AXdfvBC}{Aah, Beh, Zeh}
\nomenclature{ABdfC}{Aah, Beh, Zeh}
\nomenclature{AXdfBC}{Aah, Beh, Zeh}
\nomenclature{ABeqC}{Aah, Beh, Zeh}
\nomenclature{AXBjkC}{Aah, Beh, Zeh}
\nomenclature{ABtC}{Aah, Beh, Zeh}
\nomenclature{AtXBC}{Aah, Beh, Zeh}
\nomenclature{ABC}{Aah, Beh, Zeh}
\nomenclature{AX5BC}{Aah, Beh, Zeh}
\nomenclature{ABtC}{Aah, Beh, Zeh}
\nomenclature{AX7BC}{Aah, Beh, Zeh}
\nomenclature{ABC}{Aah, Beh, Zeh}
\nomenclature{AX5BC}{Aah, Beh, Zeh}
\nomenclature{ABCd}{Aah, Beh, Zeh}
\nomenclature{AXerztBC}{Aah, Beh, Zeh}
\nomenclature{ABtzC}{Aah, Beh, Zeh}
\nomenclature{AXBaerC}{Aah, Beh, Zeh}
\nomenclature{ABarC}{Aah, Beh, Zeh}
\nomenclature{AXrtBC}{Aah, Beh, Zeh}

%\widegraphicsRotate{VAR1-TCP-EchoService_puml_001}{fig:TCPEchoSequenceDiagram}{Sequenzdiagramm\index{Sequenzdiagramm} des iterativen TCP Servers des Echo Services  (vereinfacht)}
\lipsum[7-11]
\begin{remark}
    \ifthenelse{\boolean{print}}{
        this will be a printed book
    }{
        this is an e-book
    }
\end{remark}
\lipsum[13-16]
\activitiessection
\lipsum[17]
\begin{activity}{Aktivität\_a}{}
    \lipsum[18-19]
\end{activity}
\begin{activity}{Aktivität\_b}{nur wenn \dots}
    \lipsum[20]
\end{activity}
\lipsum[21]
\begin{activity}{Aktivität\_c}{}
    \lipsum[22]
\end{activity}

\varchapter{Name von Kapitel\_2}{\thisccnote}
\lipsum[1-4]
\section{Abschnitt}
\lipsum[5-10]\index{fde}\index{ode}\index{cpe}
\section{Abschnitt}
\lipsum[11-16]\index{cde}\index{kde}\index{öde}\index{ade}\index{12de}\index{c23de}
\index{cd4e}\index{cdwfere}
\subsection{Abschnitt}
\lipsum[17-19]
\varpart{Teil\_B}{head1.png}{Bildnachweis für head 1 (Teil B)}
\varchapter{Ein weiteres Kapitel}{\thisccnote}
\subsection{Abschnitt}
\lipsum[20-23]
\subsubsection{Unterabschnitt}
\lipsum[20-23]
\subsubsection{Unterabschnitt}
\lipsum[20-23]
\subsection{Abschnitt}
\lipsum[20-23]
\subsection{Abschnitt}
\lipsum[2340]\index{dyf}\index{dysdff}\index{dy12ef}\index{dyfwqe}\index{dyref}\index{dy3f}\index{dywerf}\index{dyfadrg}\index{dyf43}
\index{dyf}\index{dyfwerdfc<c}\index{dywtrgwrthf}\index{dyertgareqqqqqqqeeeeeeeeeeeeeeeeeeeeeeeeeeeeeeeeeef}
\subsection{Abschnitt}
\lipsum[20]
\subsection{Abschnitt}
\lipsum[20]
\subsection{Abschnitt}
\lipsum[20]
\subsection{Abschnitt}
\lipsum[20]Hier ist der Bildnachweis \q{abcdef}.\bildnachweis{abcdef}
\subsection{Abschnitt}
\lipsum[20]\index{a}\index{a1}\index{a2}\index{a3}\index{a6}\index{b3}\index{b3}\index{b5}\index{b10}\index{b4}\index{c}\index{c5}\index{c7}\index{d}\index{d7}
\subsection{Abschnitt}
\lipsum[20]Hier ist der Bildnachweis \q{xyz}.\bildnachweis{xyz, Quelle: lskdjflskj}
\subsection{Abschnitt}
\lipsum[20]
\section{Abschnitt}
\lipsum[20]
\subsection{Abschnitt}
\lipsum[20]
\end{document}