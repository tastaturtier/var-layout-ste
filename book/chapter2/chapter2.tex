\varchapter{Name von Kapitel\_2}{
    S.~Leuchter, \textit{API-Technologien und -Design}, DOI:~\href{https://doi.org/10.5281/zenodo.XXXXXXX}{10.5281/zenodo.XXXXXXX}\\
    \href{https://creativecommons.org/licenses/by-nc-sa/4.0/}{CC BY-NC-SA 4.0} 
    Steinbeis-Edition (Verteilte Architekturen; 2, Stuttgart 2020
}
\lipsum[1-4] 

\begin{excursus}{Zusatzinformationen}
    Muss auf eine Seite passen, wird als \textit{float} positioniert.
    \par\lipsum[1-2]
\end{excursus}

\lipsum[8-10]
\varwidegraphics[Angabe von Quelle und Lizenz für das Bild (2.1)]{book/chapter2/head1.png}{fig:labelname2.1}{Beschreibung \dots}
\lipsum[11-12]

\begin{remark}
    Hier sollte etwas wichtiges\index{was wichtiges} stehen.
\end{remark}
\section{Abschnitt}
\lipsum[13]
\subsection{Unterabschnitt}
\lipsum[14]
\subsubsection{Unterunterabschnitt}
\lipsum[15]
\paragraph{Unterunterunterabschnitt}
\lipsum[16]
\subparagraph{Unterunterunterunterabschnitt}
\lipsum[17]

\varactivitiessection
Hier beginnt das Praktikum mit \textit{activity}-Umgebungen\dots

\lipsum[17]

\begin{activity}{Aktivität\_a}{}
    \lipsum[18-19]
\end{activity}

\begin{activity}{Aktivität\_b}{nur wenn \dots}
    \lipsum[20]
\end{activity}

\lipsum[20]

\begin{activity}{Aktivität\_c}{}
    \lipsum[22]
\end{activity}
