  \varchapter{Name von Kapitel\_1}{
    S.~Leuchter, \textit{API-Technologien und -Design}, DOI:~\href{https://doi.org/10.5281/zenodo.XXXXXXX}{10.5281/zenodo.XXXXXXX}\\
    \href{https://creativecommons.org/licenses/by-nc-sa/4.0/}{CC BY-NC-SA 4.0} 
    Steinbeis-Edition (Verteilte Architekturen; 2, Stuttgart 2020
}
\lipsum[1-4] 

\begin{vartable}[h]{|r||l|l|}{tab:xyz}{eine Tabelle \dots}
  \hline
  \textbf{X}&\textbf{Y}&\textbf{Z}\\\hline\hline
  x&y&z\\\hline
  x&y&z\\\hline
  x&y&z\\\hline
\end{vartable}

Referenz auf die Tabelle am Seitenende: \ref{tab:test} (S.~\pageref{tab:test}).
Referenz auf die Tabelle am Seitenanfang: \ref{tab:xyz} (S.~\pageref{tab:xyz}).

\begin{vartable}[b]{|r||l|l|}{tab:test}{noch eine Tabelle}
  \hline
  \textbf{X}&\textbf{Y}&\textbf{Z}\\\hline\hline
  x&y&z\\\hline
  x&y&z\\\hline
  x&y&z\\\hline
\end{vartable}

\begin{vartable}{|r||l|l|l|l|l|l|l|l|l|l|l|l|l|l|l|l|}{tab:abc}{und noch eine weitere Tabelle}
  \hline
  \textbf{X}&\textbf{Y}&\textbf{Z}\\\hline\hline
  x&y&z&y&z&y&z&y&z&z&y&z&y&z&y&z\\\hline
  x&y&z&y&z&y&z&y&z&z&y&z&y&z&y&z\\\hline
  x&y&z&y&z&y&z&y&z&z&y&z&y&z&y&z\\\hline
\end{vartable}

\vargraphics[Bildquelle]{b}{0.5}{book/chapter1/example1.png}{fig:labelname1xx}{Beschreibung von xx\dots}

% verwendete Abkürzungen (dürfen überall im Text auftauchen)
\nomenclature{ABC}{Alfabetanfang} % Dopplungen bei identischen Abkürzungen

\begin{excursus}{Zusatzinformationen}   
    \lipsum[1-2]
\end{excursus}

\lipsum[5-7]

\vargraphicsWideRotate[Angabe von Quelle und Lizenz für das Bild (1.1)]{}{book/chapter1/example1.png}{fig:labelname1}{Beschreibung \dots}
\vargraphicsWide[Angabe von Quelle und Lizenz für das Bild (1.2)]{}{book/chapter1/example1.png}{fig:labelname2}{Beschreibung \dots}
\vargraphicsWideRotate[Angabe von Quelle und Lizenz für das Bild (1.3)]{}{book/chapter1/example1.png}{fig:labelname3}{Beschreibung \dots}


\lipsum[8-10]

\begin{remark}
    Hier sollte etwas wichtiges\index{was wichtiges} stehen.
\end{remark}
\section{Abschnitt}
\lipsum[13]
\subsection{Unterabschnitt}
\lipsum[14]
\subsubsection{Unterunterabschnitt}
\lipsum[15]
\paragraph{Unterunterunterabschnitt}
\lipsum[16]
\subparagraph{Unterunterunterunterabschnitt}
\lipsum[17]

\varactivitiessection
Hier beginnt das Praktikum mit \textit{activity}-Umgebungen\dots

\lipsum[17]

\begin{activity}{Aktivität\_a}{}
    \lipsum[18-19]
\end{activity}

\begin{activity}{Aktivität\_b}{nur wenn \dots}
    \lipsum[20]
\end{activity}

normally do not use floats for listings:
\begin{lstlisting}
import java.util.*;
public class Test extends Toast implements Brot {
    // ...
}
\end{lstlisting}

Listings can have captions, and can be references (e.\,g.~Listing~\ref{listing:htmllisting}).
\begin{lstlisting}[language=HTML, caption={sourcecode in different languages possible, can have a caption}, label=listing:htmllisting]
<html>
  <head><title>abc def</title></head>
  <body>
    <ul>
      <li>...<ul><li>...</li></ul></li>
    </ul>
  </body>
</html>
\end{lstlisting}

short listings (max 3 lines):
\begin{lstlisting}[style=nonumbers]
$ /bin/rm -r /
\end{lstlisting}
