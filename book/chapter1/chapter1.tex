\varchapter{Name von Kapitel\_1}{
    S.~Leuchter, \textit{API-Technologien und -Design}, DOI:~\href{https://doi.org/10.5281/zenodo.XXXXXXX}{10.5281/zenodo.XXXXXXX}\\
    \href{https://creativecommons.org/licenses/by-nc-sa/4.0/}{CC BY-NC-SA 4.0} 
    Steinbeis-Edition (Verteilte Architekturen; 2, Stuttgart 2020
}
\lipsum[1-4] 

\begin{excursus}{Zusatzinformationen}
    Muss auf eine Seite passen, wird als \textit{float} positioniert.
    
    \lipsum[1-2]
\end{excursus}

\lipsum[5-7]

\varwidegraphicsRotate{book/chapter1/example1.png}{fig:labelname1}{Beschreibung \dots}
\makeatletter\var@credit[fig:labelname1]{Angabe von Quelle und Lizenz für das Bild}\makeatother
\varwidegraphics{img/head2.png}{fig:labelname2}{Beschreibung \dots}
\makeatletter\var@credit{Angabe von Quelle und Lizenz für das Bild}\makeatother
\varwidegraphicsRotate{book/chapter1/example1.png}{fig:labelname3}{Beschreibung \dots}
\makeatletter\var@credit[fig:labelname3]{Angabe von Quelle und Lizenz für das Bild}\makeatother


\lipsum[8-10]

\begin{remark}
    Hier sollte etwas wichtiges\index{was wichtiges} stehen.
\end{remark}
\section{Abschnitt}
\lipsum[13]
\subsection{Unterabschnitt}
\lipsum[14]
\subsubsection{Unterunterabschnitt}
\lipsum[15]
\paragraph{Unterunterunterabschnitt}
\lipsum[16]
\subparagraph{Unterunterunterunterabschnitt}
\lipsum[17]

\varactivitiessection
Hier beginnt das Praktikum mit \textit{activity}-Umgebungen\dots

\lipsum[17]

\begin{activity}{Aktivität\_a}{}
    \lipsum[18-19]
\end{activity}

\begin{activity}{Aktivität\_b}{nur wenn \dots}
    \lipsum[20]
\end{activity}

normally do not use floats for listings:
\begin{lstlisting}%default language is Java
import java.util.*;
public class Test extends Toast implements Brot {
    // ...
}
\end{lstlisting}

Listings can have captions, and can be references (e.\,g.~Listing~\ref{listing:htmllisting}).
\begin{lstlisting}[language=HTML, caption={sourcecode in different languages possible, can have a caption}, label=listing:htmllisting]
<html>
  <head><title>abc def</title></head>
  <body>
    <ul>
      <li>...<ul><li>...</li></ul></li>
    </ul>
  </body>
</html>
\end{lstlisting}

short listings (max 3 lines):
\begin{lstlisting}[style=nonumbers]
$ /bin/rm -r /
\end{lstlisting}
